%==============================================================================
% Voorbeeld hogent-article: onderzoeksvoorstel bachproef
%==============================================================================

\documentclass{hogent-article}

\usepackage{lipsum} % Voor vultekst

% Invoegen bibliografiebestand
\addbibresource{references.bib}

% Informatie over de opleiding, het vak en soort opdracht
\studyprogramme{Professionele bachelor toegepaste informatica}
\course{Research Methods}
\assignmenttype{Paper: Onderzoeksvoorstel}
\academicyear{2022-2023} % TODO: pas het academiejaar aan

% TODO (fase 1): Werktitel
\title{Blazor en React: een vergelijkende studie voor het ontwikkelen van single page applications}

% TODO (fase 1): Studentnaam en emailadres invullen
\author{Zhao Luo (Brian)}
\email{zhao.luo@student.hogent.be}

% TODO (fase 1): Medestudent
% Schrijf je het voorstel in samenwerking met een medestudent? Geef dan de naam
% en emailadres hier. Als je het voorstel alleen schrijft, verwijder dan deze
% regels of zet ze in commentaar.
%\author{Yasmine Alaoui}
%\email{yasmine.alaoui@student.hogent.be}

% TODO (fase 1): Geef hier de link naar jullie Github-repository
\projectrepo{https://github.com/HoGentTIN/rm-2223-paper-rmluozhao}

% Binnen welke specialisatierichting uit 3TI situeert dit onderzoek zich?
% Kies uit deze lijst:
%
% - Mobile \& Enterprise development
% - AI \& Data Engineering
% - Functional \& Business Analysis
% - System \& Network Administrator
% - Mainframe Expert
% - Als het onderzoek niet past binnen een van deze domeinen specifieer je deze
%   zelf
%
\specialisation{Mobile \& Enterprise development}
% Geef hier enkele sleutelwoorden die je onderwerp beschrijven
\keywords{Single page application, Blazor, React}

\begin{document}

\begin{abstract}
  Het belang van dit onderzoek is om uitsluitsel te geven over de keuze tussen Blazor WebAssembly en React voor het ontwikkelen van e-commerce single page applications. Het doel hiervan is bedrijven die voornamelijk webshops bouwen advies geven zodat ze indien nodig kunnen overschakelen naar de technologie die meer efficiënt is. Het onderzoek zal grotendeels gerealiseerd worden door het ontwikkelen van eenzelfde e-commerce applicatie met beide technologieën en daarna de laadtijden te vergelijken. We verwachten dat React betere resultaten zal kunnen voorleggen na het experiment.
\end{abstract}

\tableofcontents

\bigskip

% TODO: Neem je dit jaar ook de bachelorproef op? Haal dan de tekst hieronder
% uit commentaar en pas het aan.

%\paragraph{Opmerking}

% Ik neem dit jaar ook de bachelorproef op. De inhoud van dit onderzoeksvoorstel dient ook als het onderwerpvoor mijn bachelorproef. Mijn promotor is (Mr./Mevr.) X.\ Familienaam.

% Beschrijf de eventuele verschillen en/of verbeteringen in dit document t.o.v.\ jouw onderzoeksvoorstel dat je ingediend hebt voor de bachelorproef.

\section{Inleiding}%
\label{sec:inleiding}

% TODO: (fase 1) introduceer je gekozen onderwerp, formuleer de onderzoeksvraag en deelvragen. Wat is de doelstelling (is die S.M.A.R.T.?), wat zal het resultaat zijn van het onderzoek (een Proof-of-Concept, een prototype, een advies, ...)? Waarom is het nuttig om dit onderwerp te onderzoeken?

Het gebruik van single page applications (SPA) is de afgelopen jaren enorm populair geworden ten koste van de traditionele webapplicatie. Het grote verschil tussen traditionele en moderne webapplicaties zit in de manier van ophalen van de pagina. Bij een SPA wordt de pagina dynamisch opgebouwd in plaats van een nieuwe pagina van de server op te vragen. De inhoud van de pagina bestaat uit meerdere kleine delen code die pas worden toegevoegd zodra dit nodig is. Het grote voordeel hiervan is dat de pagina niet herladen moet worden. Dit resulteert in een sneller werkende webapplicatie.
\newline\newline
Er bestaan verschillende technologieën voor het bouwen van SPA's. Binnen bedrijven is het echter de gewoonte om dezelfde technologie te hanteren voor alle projecten, ongeacht of dit de beste keuze is voor een specifiek project.
\newline\newline
In dit onderzoek wordt de vergelijking gemaakt tussen de front-end frameworks Blazor WebAssembly en React. We wensen te onderzoeken wat het verschil is tussen deze twee technologieën en welke het meest geschikt is voor het ontwikkelen van een e-commerce single page application. De keuze voor deze twee frameworks kwam er doordat ze elk een verschillende programmeertaal hanteren. React maakt gebruik van het gangbare JavaScript terwijl Blazor WebAssembly gebruikmaakt van C\#. De doelgroep van dit onderzoek zijn bedrijven met als kerntaak het ontwikkelen van e-commerce websites. Het resultaat zal dan ook een advies zijn voor deze bedrijven zodat zij de technologie kunnen kiezen die het meest efficiënt is.
\newline\newline
Er zal geprobeerd worden om een antwoord te vinden op de volgende onderzoeksvraag:

\begin{itemize}
    \item Welk front-end framework presteert het best voor de ontwikkeling van een e-commerce single page application, Blazor WebAssembly of React?
\end{itemize}

\section{Literatuurstudie}%
\label{sec:literatuurstudie}

% TODO: (fase 4) schrijf de literatuurstudie uit en gebruik waar gepast referenties naar de vakliteratuur.

% Refereren naar de literatuur kan met:
% \autocite{BIBTEXKEY} -> (Auteur, jaartal)
% \textcite{BIBTEXKEY} -> Auteur (jaartal)

Het concept achter een applicatie met één pagina werd al beschreven in 2002. De implementatie ervan liet echter nog enkele jaren op zich wachten. In 2006 werd AJAX (Asynchronous JavaScript and XML) geïntroduceerd. Hiermee kan een pagina nieuwe inhoud verkrijgen zonder ververst te moeten worden, wat positief is voor de gebruikerservaring.  De doorbraak van SPA's is tot stand gekomen door het toenemende belang van snelheid en UX (user experience)~\autocite{Cwik2021}.
\newline\newline
Vóór de komst van SPA's waren bedrijven gebonden aan de traditionele webapplicatie. Hierbij worden volledige pagina's opnieuw geladen van de server. Een SPA daarentegen is een webapplicatie die de huidige pagina dynamisch herschrijft met nieuwe gegevens die het ontvangt van de server. Deze werkwijze is zeer handig bij pagina's waar veel informatie hetzelfde blijft. Bij elke interactie van een gebruiker worden slechts enkele onderdelen vernieuwd door de browser. Hierdoor wordt de tijd tussen het laden van twee pagina's verkort met als gevolg dat dit veel vloeiender en aangenamer aanvoelt voor de gebruiker. Het gebruik van een moderne webapplicatie voor websites met inhoud dat veel herhaald wordt heeft een grote positieve impact ~\autocite{Lawson2022}.
\newline\newline
Blazor is een open-source framework dat werd ontwikkeld in 2018 door Microsoft. Het maakt deel uit van het .NET ecosysteem. Met Blazor is het mogelijk om webapplicaties te bouwen met C\# voor zowel de client- als servercode. Een groot voordeel is dat Blazor gebruik kan maken van elke bibliotheek dat compatibel is met het .NET systeem. Dit zorgt voor een brede waaier aan functionaliteit. Blazor WebAssembly draait op de client-side in browser. Bij het opstarten worden alle bestanden en dergelijke opgehaald die nodig zijn voor de applicatie. Eens de applicatie is opgestart, is er geen verbinding met de server meer nodig~\autocite{Spasojevic2022}. WebAssembly is een binair formaat waarmee ontwikkelaars webapplicaties kunnen bouwen in een andere programmeertaal dan JavaScript. Het werd in 2017 op de markt gebracht door W3C. WebAssembly zorgt ervoor dat de geschreven C\#-code wordt gecompileerd naar een binair formaat dat compacter is dan JavaScript. Dit formaat wordt vervolgens in de browser geladen en uitgevoerd. Dit wordt bewerkstelligd met een WebAssembly runtime omgeving ~\autocite{Yegulalp2022}.
\newline\newline
React is een JavaScript-bibliotheek dat werd geïntroduceerd in 2013 door Meta, het moederbedrijf van Facebook. React werd ontworpen met als doel het bouwen van een eenvoudige en snelle front-end voor een webapplicatie. Eén van de grootste voordelen is de mogelijkheid om componenten te hergebruiken. Dit zorgt voor een efficiëntere werking voor ontwikkelaars omdat ze code voor dezelfde functie opnieuw kunnen gebruiken. Een positief effect hiervan is dat de pagina vlotter gerenderd wordt. Dit komt mede door het gebruik van een Virtuele DOM. Het DOM (Document Object Model) is een boomgestructureerde weergave van alle elementen op een HTML-pagina. Hiermee kan de ontwikkelaar op een dynamische manier elementen op een pagina wijzigen. Met een Virtuele DOM worden alle wijzigingen eerst getest vooraleer ze worden doorgevoerd naar het werkelijke DOM~\autocite{Gundaniya2023}.
\newline\newline
Eén van de belangrijkste verschillen tussen Blazor en React is de gebruikte programmeertaal. Vóór de introductie van Blazor werden alle grote front-end frameworks geschreven in JavaScript. Het is dus niet verassend dat de komst van Blazor en het daarbij horende C\# destijds voor wat commotie zorgde in de gemeenschap van ontwikkelaars~\autocite{Davidson2022}. Het is een relatief nieuw framework dat als alternatief dient voor de ontwikkeling van SPA's zonder gebruik van JavaScript. Een groot voordeel van JavaScript-frameworks is dat ze al sterk geëvolueerd zijn en bijgevolg meer functies hebben. React en Blazor vertonen op enkele vlakken gelijkenissen met elkaar en kunnen worden gebruikt om dezelfde taken uit te voeren. Het maken van een keuze tussen de twee technologieën hangt af van meerdere factoren waaronder het type project, performantie en teamvoorkeur~\autocite{Asiuwhu2022}. 

\section{Methodologie}%{\tiny }
\label{sec:methodologie}

% TODO: (fase 5) beschrijf in detail in welke fasen je onderzoek uiteenvalt, hoe lang elke fase duurt en wat het concrete resultaat van elke fase is. Welke onderzoekstechniek ga je toepassen om elk van je onderzoeksvragen te beantwoorden? Gebruik je hiervoor experimenten, vragenlijsten, simulaties? Je beschrijft ook al welke tools je denkt hiervoor te gebruiken of te ontwikkelen.

Om het onderzoek te realiseren zal eenzelfde e-commerce single page application worden uitgewerkt met zowel Blazor WebAssembly als React. Deze applicatie zal dienen als referentie voor het vergelijken van de twee technologieën.
\newline\newline
De eerste fase van het onderzoek zal bestaan uit het ontwikkelen van de applicatie met beide technologieën. De applicatie zal aan een aantal functionaliteiten moeten voldoen:

\begin{itemize}
    \item Producten ophalen en tonen
    \item Functie voor zoeken, filteren en sorteren
    \item CRUD-operaties op producten
\end{itemize}

In de tweede fase worden de laadtijden van beide applicaties vergeleken met Google Lighthouse. De criteria waarop de laadtijden worden beoordeeld zijn o.a. first contentful paint en speed index. Ook wordt er gekeken of we erin geslaagd zijn om alle functionaliteiten te implementeren voor beide technologieën.

\section{Verwachte resultaten}%
\label{sec:verwachte-resultaten}

% TODO: (fase 6) beschrijf wat je verwacht uit je onderzoek en waarom (bv. volgens je literatuuronderzoek is softwarepakket A het meest gebruikte en denk je dat het voor deze casus ook het meest geschikt zal zijn). Natuurlijk kan je niet in de toekomst kijken en mag je geen alternatieve mogelijkheden uitsluiten. In de praktijk gebeurt het ook vaak dat een onderzoek tot verrassende resultaten leidt, dat maakt het proces nog interessanter!

Het zou mogelijk moeten zijn om elke functionaliteit uit te werken met beide frameworks. Er wordt verwacht dat na het experiment met Google Lighthouse, de laadtijden van Blazor WebAssembly langer zijn dan die van React.

\section{Conclusie}%
\label{sec:discussie-conclusie}

Na het uitvoeren van het onderzoek wordt verwacht dat de performantie bij React beter zal zijn dan bij Blazor WebAssembly. Voor de ontwikkeling van single page applications in het algemeen en een e-commerce applicatie in het bijzonder zijn JavaScript-frameworks nog steeds de standaard. De reden hiervoor is dat ze beter geëvolueerd zijn en meer functies hebben dan hun tegenhanger C\#. We verwachten dat alle opgesomde functionaliteiten met beide technologieën geïmplementeerd kunnen worden. Momenteel wordt aangenomen dat React sterker in de schoenen staat dan Blazor WebAssembly.

%------------------------------------------------------------------------------
% Referentielijst
%------------------------------------------------------------------------------
% TODO: (fase 4) de gerefereerde werken moeten in BibTeX-bestand
% bibliografie.bib voorkomen. Gebruik JabRef om je bibliografie bij te
% houden.

\printbibliography[heading=bibintoc]

\end{document}
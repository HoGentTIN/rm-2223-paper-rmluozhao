%==============================================================================
% Voorbeeld hogent-article: onderzoeksvoorstel bachproef
%==============================================================================

\documentclass{hogent-article}

\usepackage{lipsum} % Voor vultekst

% Invoegen bibliografiebestand
\addbibresource{references.bib}

% Informatie over de opleiding, het vak en soort opdracht
\studyprogramme{Professionele bachelor toegepaste informatica}
\course{Research Methods}
\assignmenttype{Paper: Onderzoeksvoorstel}
\academicyear{2022-2023} % TODO: pas het academiejaar aan

% TODO (fase 1): Werktitel
\title{Blazor en React: een vergelijkende studie voor het ontwikkelen van single page applications}

% TODO (fase 1): Studentnaam en emailadres invullen
\author{Zhao Luo (Brian)}
\email{zhao.luo@student.hogent.be}

% TODO (fase 1): Medestudent
% Schrijf je het voorstel in samenwerking met een medestudent? Geef dan de naam
% en emailadres hier. Als je het voorstel alleen schrijft, verwijder dan deze
% regels of zet ze in commentaar.
%\author{Yasmine Alaoui}
%\email{yasmine.alaoui@student.hogent.be}

% TODO (fase 1): Geef hier de link naar jullie Github-repository
\projectrepo{https://github.com/HoGentTIN/rm-2223-paper-rmluozhao}

% Binnen welke specialisatierichting uit 3TI situeert dit onderzoek zich?
% Kies uit deze lijst:
%
% - Mobile \& Enterprise development
% - AI \& Data Engineering
% - Functional \& Business Analysis
% - System \& Network Administrator
% - Mainframe Expert
% - Als het onderzoek niet past binnen een van deze domeinen specifieer je deze
%   zelf
%
\specialisation{Mobile \& Enterprise development}
% Geef hier enkele sleutelwoorden die je onderwerp beschrijven
\keywords{Single page application, Blazor, React}

\begin{document}

\begin{abstract}
  Hier neem je de abstract van je onderzoeksvoorstel over.
\end{abstract}

\tableofcontents

\bigskip

% TODO: Neem je dit jaar ook de bachelorproef op? Haal dan de tekst hieronder
% uit commentaar en pas het aan.

%\paragraph{Opmerking}

% Ik neem dit jaar ook de bachelorproef op. De inhoud van dit onderzoeksvoorstel dient ook als het onderwerpvoor mijn bachelorproef. Mijn promotor is (Mr./Mevr.) X.\ Familienaam.

% Beschrijf de eventuele verschillen en/of verbeteringen in dit document t.o.v.\ jouw onderzoeksvoorstel dat je ingediend hebt voor de bachelorproef.

\section{Inleiding}%
\label{sec:inleiding}

% TODO: (fase 1) introduceer je gekozen onderwerp, formuleer de onderzoeksvraag en deelvragen. Wat is de doelstelling (is die S.M.A.R.T.?), wat zal het resultaat zijn van het onderzoek (een Proof-of-Concept, een prototype, een advies, ...)? Waarom is het nuttig om dit onderwerp te onderzoeken?

Het gebruik van single page applications (SPA) is de afgelopen jaren enorm populair geworden ten koste van de traditionele webapplicatie. Het grote verschil tussen traditionele en moderne webapplicaties zit in de manier van ophalen van de pagina. Bij een SPA wordt de pagina dynamisch opgebouwd in plaats van een nieuwe pagina van de server op te vragen. De inhoud van de pagina bestaat uit meerdere kleine delen code die pas worden toegevoegd zodra dit nodig is. Het grote voordeel hiervan is dat de pagina niet herladen moet worden. Dit resulteert in een sneller werkende webapplicatie.
\newline\newline
Er bestaan verschillende technologieën voor het bouwen van SPA's. Binnen bedrijven is het echter de gewoonte om dezelfde technologie te hanteren voor alle projecten, ongeacht of dit de beste keuze is voor een specifiek project.
\newline\newline
In dit onderzoek wordt de vergelijking gemaakt tussen de front-end frameworks Blazor WebAssembly en React. We wensen te onderzoeken wat het verschil is tussen deze twee technologieën en welke het meest geschikt is voor het ontwikkelen van een e-commerce single page application. De keuze voor deze twee frameworks kwam er doordat ze elk een verschillende programmeertaal hanteren. React maakt gebruik van het gangbare JavaScript terwijl Blazor WebAssembly gebruikmaakt van C\#. De doelgroep van dit onderzoek zijn bedrijven met als kerntaak het ontwikkelen van e-commerce websites. Het resultaat zal dan ook een advies zijn voor deze bedrijven zodat zij de technologie kunnen kiezen die het meest efficiënt is.
\newline\newline
Er zal geprobeerd worden om een antwoord te vinden op de volgende onderzoeksvraag:

\begin{itemize}
    \item Welk front-end framework presteert het best voor de ontwikkeling van een e-commerce single page application, Blazor WebAssembly of React?
\end{itemize}

\section{Literatuurstudie}%
\label{sec:literatuurstudie}

% TODO: (fase 4) schrijf de literatuurstudie uit en gebruik waar gepast referenties naar de vakliteratuur.

% Refereren naar de literatuur kan met:
% \autocite{BIBTEXKEY} -> (Auteur, jaartal)
% \textcite{BIBTEXKEY} -> Auteur (jaartal)
Voorbeeld van een referentie waar de auteursnaam geen onderdeel van de zin is~\autocite{Moore2002}.

\lipsum[4-9]

\section{Methodologie}%
\label{sec:methodologie}

% TODO: (fase 5) beschrijf in detail in welke fasen je onderzoek uiteenvalt, hoe lang elke fase duurt en wat het concrete resultaat van elke fase is. Welke onderzoekstechniek ga je toepassen om elk van je onderzoeksvragen te beantwoorden? Gebruik je hiervoor experimenten, vragenlijsten, simulaties? Je beschrijft ook al welke tools je denkt hiervoor te gebruiken of te ontwikkelen.

\lipsum[10-12]

\section{Verwachte resultaten}%
\label{sec:verwachte-resultaten}

% TODO: (fase 6) beschrijf wat je verwacht uit je onderzoek en waarom (bv. volgens je literatuuronderzoek is softwarepakket A het meest gebruikte en denk je dat het voor deze casus ook het meest geschikt zal zijn). Natuurlijk kan je niet in de toekomst kijken en mag je geen alternatieve mogelijkheden uitsluiten. In de praktijk gebeurt het ook vaak dat een onderzoek tot verrassende resultaten leidt, dat maakt het proces nog interessanter!

\lipsum[14-18]

\section{Discussie, conclusie}%
\label{sec:discussie-conclusie}

\lipsum[19-21]

%------------------------------------------------------------------------------
% Referentielijst
%------------------------------------------------------------------------------
% TODO: (fase 4) de gerefereerde werken moeten in BibTeX-bestand
% bibliografie.bib voorkomen. Gebruik JabRef om je bibliografie bij te
% houden.

\printbibliography[heading=bibintoc]

\end{document}